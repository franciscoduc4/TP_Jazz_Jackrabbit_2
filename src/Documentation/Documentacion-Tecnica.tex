\documentclass[titlepage,a4paper]{article}
\usepackage{a4wide}
\usepackage[T1]{fontenc}
\usepackage{lipsum}
\usepackage{enumitem,amssymb}
\newlist{todolist}{itemize}{2}
\setlist[todolist]{label=$\square$}
\usepackage{pifont}
\usepackage{xcolor}
\usepackage[normalem]{ulem}
\usepackage{listings}
\usepackage{color}
\usepackage{pdfpages}
\usepackage{multirow}
\usepackage{plantuml}
\usepackage{fancyhdr}
\usepackage[spanish,es-tabla]{babel}

% Quote box
% for adjustwidth environment
\usepackage[strict]{changepage}

% for formal definitions
\usepackage{framed}

% Add the titlesec package for optional formatting (if needed)
\usepackage{titlesec}

\newcounter{subsubsubsection}[subsubsection]
\renewcommand\thesubsubsubsection{\thesubsubsection.\arabic{subsubsubsection}}
\renewcommand\theparagraph{\thesubsubsubsection.\arabic{paragraph}} % optional; useful if paragraphs are to be numbered

\titleformat{\subsubsubsection}
  {\normalfont\normalsize\bfseries}{\thesubsubsubsection}{1em}{}
\titlespacing*{\subsubsubsection}
{0pt}{3.25ex plus 1ex minus .2ex}{1.5ex plus .2ex}

\makeatletter
\renewcommand\paragraph{\@startsection{paragraph}{5}{\z@}%
  {3.25ex \@plus1ex \@minus.2ex}%
  {-1em}%
  {\normalfont\normalsize\bfseries}}
\renewcommand\subparagraph{\@startsection{subparagraph}{6}{\parindent}%
  {3.25ex \@plus1ex \@minus .2ex}%
  {-1em}%
  {\normalfont\normalsize\bfseries}}
\def\toclevel@subsubsubsection{4}
\def\toclevel@paragraph{5}
\def\toclevel@paragraph{6}
\def\l@subsubsubsection{\@dottedtocline{4}{7em}{4em}}
\def\l@paragraph{\@dottedtocline{5}{10em}{5em}}
\def\l@subparagraph{\@dottedtocline{6}{14em}{6em}}
\makeatother

\setcounter{secnumdepth}{5}
\setcounter{tocdepth}{4}


% environment derived from framed.sty: see leftbar environment definition
\definecolor{formalshade}{rgb}{0.95,0.95,1}

\newenvironment{formal}{%
  \def\FrameCommand{%
    \hspace{1pt}%
    {\color{blue}\vrule width 2pt}%
    {\color{formalshade}\vrule width 4pt}%
    \colorbox{formalshade}%
  }%
  \MakeFramed{\advance\hsize-\width\FrameRestore}%
  \noindent\hspace{-4.55pt}% disable indenting first paragraph
  \begin{adjustwidth}{}{7pt}\itshape%
  \vspace{2pt}\vspace{2pt}%
}
{%
  \vspace{2pt}\end{adjustwidth}\endMakeFramed%
}

\definecolor{dkgreen}{rgb}{0,0.6,0}
\definecolor{gray}{rgb}{0.5,0.5,0.5}
\definecolor{mauve}{rgb}{0.58,0,0.82}

\lstset{frame=tb,
  language=Java,
  aboveskip=3mm,
  belowskip=3mm,
  showstringspaces=false,
  columns=flexible,
  basicstyle={\small\ttfamily},
  numbers=none,
  numberstyle=\tiny\color{gray},
  keywordstyle=\color{blue},
  commentstyle=\color{dkgreen},
  stringstyle=\color{mauve},
  breaklines=true,
  breakatwhitespace=true,
  tabsize=3,
  literate=
  {á}{{\'a}}1 {é}{{\'e}}1 {í}{{\'i}}1 {ó}{{\'o}}1 {ú}{{\'u}}1
  {Á}{{\'A}}1 {É}{{\'E}}1 {Í}{{\'I}}1 {Ó}{{\'O}}1 {Ú}{{\'U}}1
  {à}{{\`a}}1 {è}{{\`e}}1 {ì}{{\`i}}1 {ò}{{\`o}}1 {ù}{{\`u}}1
  {À}{{\`A}}1 {È}{{\'E}}1 {Ì}{{\`I}}1 {Ò}{{\`O}}1 {Ù}{{\`U}}1
  {ä}{{\"a}}1 {ë}{{\"e}}1 {ï}{{\"i}}1 {ö}{{\"o}}1 {ü}{{\"u}}1
  {Ä}{{\"A}}1 {Ë}{{\"E}}1 {Ï}{{\"I}}1 {Ö}{{\"O}}1 {Ü}{{\"U}}1
  {â}{{\^a}}1 {ê}{{\^e}}1 {î}{{\^i}}1 {ô}{{\^o}}1 {û}{{\^u}}1
  {Â}{{\^A}}1 {Ê}{{\^E}}1 {Î}{{\^I}}1 {Ô}{{\^O}}1 {Û}{{\^U}}1
  {ã}{{\~a}}1 {ẽ}{{\~e}}1 {ĩ}{{\~i}}1 {õ}{{\~o}}1 {ũ}{{\~u}}1
  {Ã}{{\~A}}1 {Ẽ}{{\~E}}1 {Ĩ}{{\~I}}1 {Õ}{{\~O}}1 {Ũ}{{\~U}}1
  {œ}{{\oe}}1 {Œ}{{\OE}}1 {æ}{{\ae}}1 {Æ}{{\AE}}1 {ß}{{\ss}}1
  {ű}{{\H{u}}}1 {Ű}{{\H{U}}}1 {ő}{{\H{o}}}1 {Ő}{{\H{O}}}1
  {ç}{{\c c}}1 {Ç}{{\c C}}1 {ø}{{\o}}1 {å}{{\r a}}1 {Å}{{\r A}}1
  {€}{{\euro}}1 {£}{{\pounds}}1 {«}{{\guillemotleft}}1
  {»}{{\guillemotright}}1 {ñ}{{\~n}}1 {Ñ}{{\~N}}1 {¿}{{?`}}1 {¡}{{!`}}1
}

\pagestyle{fancy} % Encabezado y pie de página
\fancyhf{}
\fancyhead[L]{Trabajo Práctico Final | Jazz Jackrabbit 2 | Grupo 5\\Manual de Proyecto}
\fancyhead[R]{Taller de Programación I - FIUBA}
\renewcommand{\headrulewidth}{0.4pt}
\fancyfoot[C]{\thepage}
\renewcommand{\footrulewidth}{0.4pt}

\fancypagestyle{firstPage}{%
  \fancyfoot[C]{\thepage}
  \renewcommand{\footrulewidth}{0.4pt}
}

\begin{document}

\begin{titlepage} % Carátula
	\hfill\includegraphics[width=6cm]{logofiuba.jpg}
    \centering
    \vfill
    \Huge \textbf{Trabajo Práctico Final\\Jazz Jackrabbit 2}
    \vskip2cm
    \Large [75.42] Taller de Programación I\\
    Primer Cuatrimestre de 2024
    \vfill
    \begin{tabular}{ | l | l | l | } % Datos del alumno
      \hline
      \textbf{Estudiante} & \textbf{Padrón} & \textbf{Email} \\ \hline
      Buono, Fernando & 103523 & fbuono@fi.uba.ar \\ \hline
      Duca, Francisco & 106308 & fduca@fi.uba.ar \\ \hline
      Oshiro, Lucas & 107024 & loshiro@fi.uba.ar \\ \hline
      Shiao, Tomás Jorge & 106099 & tshiao@fi.uba.ar \\ \hline
  	\end{tabular}
    \vfill
    \vfill
\end{titlepage}

\clearpage\pagestyle{empty}
\tableofcontents % Índice general
\newpage
\setcounter{page}{1}
\pagestyle{fancy}
\setcounter{secnumdepth}{5}
\setcounter{tocdepth}{5}
\section{Introducción}
El proyecto es una implementación del juego Jazz Jackrabbit 2 con soporte multijugador. Los jugadores pueden seleccionar uno de los tres personajes: Jazz, Spaz o Lori, cada uno con habilidades y ataques especiales únicos. El juego incluye la capacidad de disparar, correr, saltar y realizar ataques especiales, así como varios estados como intoxicado, recibir daño y muerte. Además, se incluye un editor de niveles para que se puedan crear nuevos niveles.

\section{Estructura del Proyecto}
El proyecto tiene una estructura de cliente servidor, donde el servidor es el encargado de manejar la lógica del juego y el cliente es el encargado de mostrar la información al usuario. Además, se incluye un editor de niveles que permite crear nuevos niveles para el juego.

\subsection{Servidor}


\subsection{Cliente}


\subsection{Editor de Niveles}


\section{Protocolo}


\section{Diagramas de Clases}


\section{Compilación y Ejecución}

\end{document}